% Sample file for Beamer Class
% Based on example 3 at http://www2.informatik.uni-freiburg.de/~frank/ENG/latex-course/latex-course-3/latex-course-3_en.html
% 澍法雨 http://ryogan.org/blog
% Text Processing using upLaTeX

\documentclass[dvipdfmx]{beamer}
\usepackage{pxjahyper} % cf. http://d.hatena.ne.jp/zrbabbler/20120528/1338221936
%\usepackage{beamerthemeshadow}
\setbeamertemplate{navigation symbols}{} 
%%% 以下は beamer とは直接関係ない

\usepackage{color}
\usepackage{ulem}
\usepackage{comment}
\usetheme{CambridgeUS}  

\begin{document}
\title[Basye]{ベイズ推定} 
\author[naga]{長島 貴之}            %% ここに書かれた情報は色々なところに使われるrdcrdc
%\institute[rdc]{rdc}   %% なるべく設定した方が良い
\date{\today}

 
\frame{\titlepage}

\frame{\frametitle{目次}\tableofcontents}
\section{ベイズと頻出}
\section{ベイズ理論のための確率論入門} 

\subsection{確率の定義}
\frame{\frametitle{確率の定義}
っっっ
}
\subsection{確率の記号}
\frame{\frametitle{確率の記号}
\ P(A) \ : 事象Aの起こるか確率
}
\subsection{確率の記号}
\frame{\frametitle{確率の記号}
2つの事象A,Bがあり,同時に起こる事象  
この事象が起こる確率  $ A  \cap B $
 \\
この事象が起こる確率 
\[
  P(A \cap B) or P(A,B)
\]
と記される(本によってまちまちだが後者多い)
}

\subsection{条件付きの確率}
\frame{\frametitle{条件付きの確率}
 ある事象Aがおこった元で事象Bが起こる確率$ P(A  \mid  B) $\\
 例題\\
 1組のトランプから1枚抜くとする.抜いたカードがハートである事象をA,絵札である事象Bとしたとき,条件付き確率$ P(A  \mid  B) $ ,$ P(B  \mid A ) $ ,同時確率$  P(A \cap B) $を求めよ.
 }
 
 \frame{\frametitle{条件付きの確率2}
  $ P(A  \mid  B) $  抜いたカードが$ \heartsuit $のとき,それが絵札である確率\\
   $ \heartsuit $は13通りで,その中で絵柄3枚なので $ \frac {3}  {13} $\\
  $ P(B  \mid  A) $  抜いたカードが絵札のとき,それが $ \heartsuit $である確率\\
  絵札が12通りで,その中で$ \heartsuit $は3枚なので$ \frac{1}{4}$\\
  $  P(A \cap B) $  抜いたカードが$ \heartsuit $かつ絵札の確率=$ \frac {3}  {52} $ = $\frac{1}{14}$\\
  }
 \frame{\frametitle{条件付きの確率2} 
  条件付き確率は次の式で表現できる.
 \begin{equation}
   P(B  \mid  A) = \frac{ P(A \cap B)}{P(A)} 
 \end{equation} \\
 例題\\
 1組のトランプから1枚抜くとする.抜いたカードがハートである事象をA,絵札である事象Bとしたとき,条件付き確率$ P(A  \mid  B) $ ,$ P(B  \mid A ) $ を求めよ.\\
 P(A)=$ \frac {13}  {52}$,P(B)=$\frac {12}  {52}$, 先のページより $  P(A \cap B) $ = $   \frac {3}  {52} $ 
 \begin{equation}
    P(B  \mid  A) = \frac{ P(A \cap B)}{P(A)} = \frac{\frac {3}  {52}}{\frac {13}  {52}}=\frac {3}  {13}
  \end{equation} \\
   ※$P(A  \mid B )$も同様なので省略
 }
 \subsection{確率の乗法定理}
 \frame{\frametitle{確率の乗法定理}
 \begin{equation}
    P(B  \mid  A) = \frac{ P(A \cap B)}{P(A)} 
  \end{equation} \\
  両辺にP(A)を掛けて
   \begin{equation}
     P(A \cap B) =P(B  \mid  A)P(A) 
    \end{equation} \\
  }
 %独立書くかも
\section{ベイズの定理} 
\subsection{導出}
\frame{\frametitle{導出}
\begin{equation}
     P(A \cap B) =P(B  \mid  A)P(A) 
  \end{equation} \\
\begin{equation}
     P(A \cap B) =P(A  \mid  B)P(B) 
\end{equation} \\  
以上 より$ P(A  \mid  B) $について解くと
 \begin{equation}
   P(A\mid  B) = \frac{ P(B \mid A P(A)}{P(B)} 
 \end{equation} \\
これが ベイズの定理.導出は意外と簡単.
%簡単な例題
}

\frame{\frametitle{変更}
前のページのままだとイメージしづらいので,少し書き換える.
\begin{equation}
   P(H\mid  D) = \frac{ P(D \mid H P(H)}{P(D)} 
 \end{equation} \\
 Aを原因と仮定(hypotheis),BをAのもとで得られる結果やデータとして置き換えた.
 }
\subsection{例題}
	 \frame{\frametitle{例題1}
	 男10女7が一室でパーリーを開いた.男子の喫煙者は5,女3である.部屋に入ったら吸い殻が一本,灰皿の上にあった.この時のタバコを吸った人が女性である確率を求めよ.\\ \pause 
	 \vspace{1zh}
	 H:女性である 	\pause  \\
	 D:(タバコを吸った人すなわち)喫煙者である.
	  }

	 \frame{\frametitle{例題1}
	 P(H)はパーティーの中で女性である確率,P(D)は喫煙者である確率,$P(D\mid H)$は女性の中で喫煙者である確率\\  \pause
	 P(H)=$\frac{7}{17}$,P(D)=$\frac{8}{17}$,$P(D\mid H)$ =$\frac{3}{7}$\\
	 式?に代入すると
	 \begin{equation}
	    P(H\mid  D) = \frac{ \frac{7}{17}\times\frac{3}{7}}{\frac{8}{17}} = \frac{3}{8}
	  \end{equation} \\
	  タバコを吸った人が女性である確率=$ \frac{3}{8} $
	  この図はベン図から容易に求められる.
 }
 
  \frame{\frametitle{ベイズの展開公式}
データDを得ることができる原因だが普通1つではない.原因がn個あれば $H_1$,$H_2$...$H_n$ とかける.
原因$H_1$に注目してみる.ベイズの定理のHを$H_1$とき置き換える.

\begin{equation}
    P(H_1\mid  D) = \frac{ P(D \mid H_1 )P(H_1)}{P(D)}
 \end{equation} \\
P(D)は同時確率の和で表現できるので
\begin{equation}
   P(D) = P(D, H_1) + P(D, H_2) + \cdots + P(D, H_n)
 \end{equation} \\
 と書ける.
 %図はホワイトボードで書ける
     
}

  \frame{\frametitle{ベイズの展開公式2}
 乗法の定理より
 \begin{equation}
     %P(D) = P(D \mid H_1 )P(H_1)+P(D \mid H_2 )P(H_2 ) +....+P(D \mid H_n )P(H_n)
     P(D) = P(D|H_1)P(H_1) + P(D|H_2)P(H_2) + \cdots + P(D|H_n)P(H_n)
  \end{equation} \\
  (乗法定理より)
  これを一般化すると
   \begin{equation}
       %P(H_i|D) = \frac{P(D|H_i)P(H_i)}{P(D|H_1)P(H_1) + P(D|H_2)P(H_2) + \dots + P(D|H_n)P(H_n)}
        P(H_i|D) = \frac{P(D|H_i)P(H_i)}{\sum _{i=0}^{n}P\left( D|H_{i}\right) P\left(H_{i}\right)}
    \end{equation} \\
}
\frame{\frametitle{用語の確認}
\begin{itemize}
\item P(H|D)
\item P(D|H\pause 
\item P(Hu)\pause 
\item 涅槃寂静
\end{itemize}

}



\frame{\frametitle{四法印 (with pause)}
\begin{itemize}
	\item 諸行無常
\item 諸法無我\pause 
\item 一切皆苦\pause 
\item 涅槃寂静
\end{itemize}
}


\subsection{番号付きのリスト}
\frame{\frametitle{四法印 (with number)}
\begin{enumerate}
\item 諸行無常
\item 諸法無我
\item 一切皆苦
\item 涅槃寂静
\end{enumerate}
}


\frame{\frametitle{四法印 (with number and pause)}
\begin{enumerate}
\item 諸行無常\pause 
\item 諸法無我\pause 
\item 一切皆苦\pause 
\item 涅槃寂静
\end{enumerate}
}

\section{セクション 3} 
\subsection{テーブル}
\frame{\frametitle{三毒}
\begin{tabular}{|c|c|c|}
\hline
\textbf{三毒} & \textbf{Sanskrit} & \textbf{Pāli} \\
\hline
貪 & rāga & lobha  \\
\hline
瞋 & dveṣa & dosa \\
\hline
癡 & moha & moha \\
\hline
\end{tabular}}


\frame{\frametitle{ 三毒 (with pause)}
\begin{tabular}{c c c}
貪 & rāga & lobha  \\
\pause 
瞋 & dveṣa & dosa \\
\pause 
癡 & moha & moha \\
\end{tabular} }


\section{セクション 4}
\subsection{ブロック}
\frame{\frametitle{三身 (with pause)}

\begin{block}{法身}
真如そのもの。代表 = 毘盧遮那仏
これは下線に\textcolor{red}{\underline{\color{black}{赤色}}}をつけます。\pause 
\end{block}\pause

\begin{exampleblock}{報身}
真理のはたらき。あるいは修行して成仏する姿。代表 = 阿弥陀仏
\end{exampleblock}\pause


\begin{alertblock}{応身}
人々の前に現れる釈迦の姿。代表 = 釈迦牟尼仏
\end{alertblock}\pause
}

\end{document}
