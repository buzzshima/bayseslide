

\documentclass[dvipdfmx]{beamer}
\usepackage{pxjahyper} 
\setbeamertemplate{navigation symbols}{} 


\usepackage{color}
\usepackage{ulem}
\usepackage{comment}
\useoutertheme{infolines}                                                                                              
\usecolortheme{dolphin}
%\usetheme{Boadilla}
\setbeamertemplate{items}[default]

\begin{document}
\title[Basye]{ベイズ推定} 
\author[naga]{長島 貴之}            %% ここに書かれた情報は色々なところに使われるrdcrdc
%\institute[rdc]{rdc}   %% なるべく設定した方が良い
\date{\today}

 
\frame{\titlepage}

\frame{\frametitle{目次}\tableofcontents}
\section{ベイズと頻出}
\section{ベイズ理論のための確率論入門} 

\subsection{確率の定義}
\frame{\frametitle{確率の定義}
っっっ
}
\subsection{確率の記号}
\frame{\frametitle{確率の記号}
\ P(A) \ : 事象Aの起こるか確率
}
\subsection{確率の記号}
\frame{\frametitle{確率の記号}
2つの事象A,Bがあり,同時に起こる事象  
この事象が起こる確率  $ A  \cap B $
 \\
この事象が起こる確率 
\[
  P(A \cap B) or P(A,B)
\]
と記される(本によってまちまちだが後者多い)
}

\subsection{条件付きの確率}
\frame{\frametitle{条件付きの確率}
 ある事象Aがおこった元で事象Bが起こる確率$ P(A  \mid  B) $\\
 例題\\
 1組のトランプから1枚抜くとする.抜いたカードがハートである事象をA,絵札である事象Bとしたとき,条件付き確率$ P(A  \mid  B) $ ,$ P(B  \mid A ) $ ,同時確率$  P(A \cap B) $を求めよ.
 }
 
 \frame{\frametitle{条件付きの確率2}
  $ P(A  \mid  B) $  抜いたカードが$ \heartsuit $のとき,それが絵札である確率\\
   $ \heartsuit $は13通りで,その中で絵柄3枚なので $ \frac {3}  {13} $\\
  $ P(B  \mid  A) $  抜いたカードが絵札のとき,それが $ \heartsuit $である確率\\
  絵札が12通りで,その中で$ \heartsuit $は3枚なので$ \frac{1}{4}$\\
  $  P(A \cap B) $  抜いたカードが$ \heartsuit $かつ絵札の確率=$ \frac {3}  {52} $ = $\frac{1}{14}$\\
  }
 \frame{\frametitle{条件付きの確率2} 
  条件付き確率は次の式で表現できる.
 \begin{equation}
   P(B  \mid  A) = \frac{ P(A \cap B)}{P(A)} 
 \end{equation} \\
 例題\\
 1組のトランプから1枚抜くとする.抜いたカードがハートである事象をA,絵札である事象Bとしたとき,条件付き確率$ P(A  \mid  B) $ ,$ P(B  \mid A ) $ を求めよ.\\
 P(A)=$ \frac {13}  {52}$,P(B)=$\frac {12}  {52}$, 先のページより $  P(A \cap B) $ = $   \frac {3}  {52} $ 
 \begin{equation}
    P(B  \mid  A) = \frac{ P(A \cap B)}{P(A)} = \frac{\frac {3}  {52}}{\frac {13}  {52}}=\frac {3}  {13}
  \end{equation} \\
   ※$P(A  \mid B )$も同様なので省略
 }
 \subsection{確率の乗法定理}
 \frame{\frametitle{確率の乗法定理}
 \begin{equation}
    P(B  \mid  A) = \frac{ P(A \cap B)}{P(A)} 
  \end{equation} \\
  両辺にP(A)を掛けて
   \begin{equation}
     P(A \cap B) =P(B  \mid  A)P(A) 
    \end{equation} \\
  }
 %独立書くかも
\section{ベイズの定理} 
\subsection{導出}
\frame{\frametitle{導出}
\begin{equation}
     P(A \cap B) =P(B  \mid  A)P(A) 
  \end{equation} \\
\begin{equation}
     P(A \cap B) =P(A  \mid  B)P(B) 
\end{equation} \\  
以上 より$ P(A  \mid  B) $について解くと
 \begin{equation}
   P(A\mid  B) = \frac{ P(B \mid A P(A)}{P(B)} 
 \end{equation} \\
これが ベイズの定理.導出は意外と簡単.
}

\frame{\frametitle{変更}
前のページのままだとイメージしづらいので,少し書き換える.
\begin{equation}
   P(H\mid  D) = \frac{ P(D \mid H P(H)}{P(D)} 
 \end{equation} \\
 Aを原因と仮定(hypotheis),BをAのもとで得られる結果やデータとして置き換えた.
 }
\subsection{例題}
	 \frame{\frametitle{例題1}
	 男10女7が一室でパーリーを開いた.男子の喫煙者は5,女3である.部屋に入ったら吸い殻が一本,灰皿の上にあった.この時のタバコを吸った人が女性である確率を求めよ.\\ \pause 
	 \vspace{1zh}
	 H:女性である 	\pause  \\
	 D:(タバコを吸った人すなわち)喫煙者である.
	  }

	 \frame{\frametitle{例題1}
	 P(H)はパーティーの中で女性である確率,P(D)は喫煙者である確率,$P(D\mid H)$は女性の中で喫煙者である確率\\  \pause
	 P(H)=$\frac{7}{17}$,P(D)=$\frac{8}{17}$,$P(D\mid H)$ =$\frac{3}{7}$\\
	 式?に代入すると
	 \begin{equation}
	    P(H\mid  D) = \frac{ \frac{7}{17}\times\frac{3}{7}}{\frac{8}{17}} = \frac{3}{8}
	  \end{equation} \\
	  タバコを吸った人が女性である確率=$ \frac{3}{8} $
	  この図はベン図から容易に求められる.
 }
 
  \frame{\frametitle{ベイズの展開公式}
データDを得ることができる原因だが普通1つではない.原因がn個あれば $H_1$,$H_2$...$H_n$ とかける.
原因$H_1$に注目してみる.ベイズの定理のHを$H_1$とき置き換える.

\begin{equation}
    P(H_1\mid  D) = \frac{ P(D \mid H_1 )P(H_1)}{P(D)}
 \end{equation} \\
P(D)は同時確率の和で表現できるので
\begin{equation}
   P(D) = P(D, H_1) + P(D, H_2) + \cdots + P(D, H_n)
 \end{equation} \\
 と書ける.
 %図はホワイトボードで書ける
     
}

  \frame{\frametitle{ベイズの展開公式2}
 乗法の定理より
 \begin{equation}
     %P(D) = P(D \mid H_1 )P(H_1)+P(D \mid H_2 )P(H_2 ) +....+P(D \mid H_n )P(H_n)
     P(D) = P(D|H_1)P(H_1) + P(D|H_2)P(H_2) + \cdots + P(D|H_n)P(H_n)
  \end{equation} \\
  (乗法定理より)
  これを一般化すると
   \begin{equation}
       %P(H_i|D) = \frac{P(D|H_i)P(H_i)}{P(D|H_1)P(H_1) + P(D|H_2)P(H_2) + \dots + P(D|H_n)P(H_n)}
        P(H_i|D) = \frac{P(D|H_i)P(H_i)}{\sum _{i=0}^{n}P\left( D|H_{i}\right) P\left(H_{i}\right)}
    \end{equation} \\
}
\frame{\frametitle{用語の確認}
\begin{itemize}
\item $P(H|D)$:事後確率...データDが原因$h_i$ から得られる確率
\item $P(D|H)$:尤度...原因 $h_i$のもとでデータDが得られる確率\pause 
\item P(H):事前確率...データDを得る前の原因$h_i$の確かしらさ\pause
\end{itemize}
}

\frame{\frametitle{ ベイズ理論の計算ステップ}%順番上げる
\begin{enumerate}
 \item モデル化し,尤度を導出
 \item 事前確率を設定
 \item ベイズに公式を用いて事後確率を算出
\end{enumerate}
}
\frame{\frametitle{理由不十分の原則}
形が同じの2つの壺,赤い壺,青い壺,が置いてある.
赤い壺には白玉2個,赤玉3個が入っていて,青い壺には,白玉4個,赤玉8個入っている.
壺を一つを選択し,玉を一つを取り出したら,白玉だった.このときの赤い壺から取り出された確率を求めよ.\\
 $H_r$:赤い壺から取り出す	\pause  \\
 $H_b$:青い壺から取り出す 	\pause  \\
 D:取り出した玉が白である.
}

\frame{\frametitle{理由不十分の原則}
べいずの展開公式
	\begin{equation}
			P(H_r|D) = \frac{P(D|H_r)P(H_r)}{P(D|H_r)P(H_r) + P(D|H_b)P(H_b) }					     
	 \end{equation} \\
尤度を算出:赤い壺から取り出された玉が白である確率 \\
$P(H_r|D) $= $\frac{2}{5}$ \\
事前確率の設定 \\
2つの壺がどの割合で割合で選ばれるかの乗法がないので,事前確率は等確率になるように設定
$P(H_r) $=$P(H_b) $=$\frac{1}{2}$\\
\newpage
今までのことを踏まえて
	\begin{equation}
			P(H_r|D) = 	\dfrac {\dfrac {2} {5}\times \dfrac {1} {2}} {\dfrac {2} {5}\times \dfrac {1} {2}+\dfrac {1} {3}\times \dfrac {1} {2}}=\dfrac{6}{11}
	 \end{equation} 
}

\frame{\frametitle{理由不十分の原則}
あとで緑の本を見てかく
}


\frame{\frametitle{ベイズ更新}
A社の壺には,水晶玉とガラス玉が4:1の割合で入っている.
B社の壺には,水晶玉とガラス玉が2:3の割合で入っている.壺の見た目は全く同じで壺にはたくさん玉が入っているとする.
いま,A社製かB社製か不明の壺があり,続けて3回玉を取り出したら,純に 水晶玉,水晶玉,ガラス玉であった.この壺がA社製の確率を求めよ.
$H_a$:A社製の壺から取り出す\\
$H_b$:B社製の壺から取り出す 	  \\
$D_1$=S:取り出した玉が水晶玉 \\
$D_2$=G:取り出した玉がガラス玉\\
}
\frame{\frametitle{ベイズ更新2}
データは3つの事象が連続になっている.またそれを取り込むためには??
$\Rightarrow$ ベイズ更新
 今回求めるものは
	\begin{eqnarray}
		  	P(H_a|S) = \frac{P(S|H_a)P(H_a)}{P(S|H_a)P(H_a) + P(S|H_b)P(H_b) }\\
		  	P(H_b|S) = \frac{P(S|H_b)P(H_b)}{P(S|H_b)P(H_b) + P(S|H_b)P(H_b) }	\\
		  	P(H_a|G) = \frac{P(G|H_a)P(H_a)}{P(G|H_a)P(H_a) + P(G|H_a)P(H_a) }	\\
		  	P(H_b|G) = \frac{P(G|H_b)P(H_b)}{P(G|H_b)P(H_b) + P(G|H_b)P(H_b) }				
		\end{eqnarray}
}
\frame{\frametitle{ベイズ更新2}
尤度の算出
\[
P(S|H_a)= 0.8=4/5  ,  P(G|H_a)= 0.2 , P(S|H_b)=0.4 =2/5 , P(G|H_b)=0.6
\]
一回目の玉の取り出し\\
事前確率の設定\\
2つの壺の選択各率は不明なので,理由不十分の原則をもちいて,
\[
P(H_a)=P(H_b)=0.5
\] \\
一回目は水晶玉名尾で
\[
P(|H_a)|S)= 2/3  ,  P(H_b|S)= 1/3 
\]
}
\frame{\frametitle{ベイズ更新2}
二回目の取り出し\\
二回目の取り出しの取り出しときのどうするか?ここでベイズ更新を使う.
一回目の事後確率を二回目の分析する際の事前確率として使う.
(ベイズ更新の使う理由を説明)
\[
P(H_a)=2/3 , P(H_b)=1/3  
\] \\
二回目の事後確率は
\[
P(H_a|S)=0.8 ,P(H_b|S)=0.2
\]
となる.
}

\frame{\frametitle{ベイズ更新2}
三回目の取り出しも同様に,
\[
P(H_a)=0.8 , P(H_b)=0.2  
\] \\
三回目の事後確率は
\[
P(H_a|G)= 4/7\fallingdotseq(0.571)
\]
となる.\\
%理由不十分の原則からA社製の壺である確率は0.5なので,確率は高まった
}
\section{ベイズ統計学}







\end{document}
