\documentclass[11pt,a4paper]{jsarticle}
%
\usepackage{amsmath,amssymb}
\usepackage{bm}
\usepackage{graphicx}
\usepackage{ascmac}
%
\setlength{\textwidth}{\fullwidth}
\setlength{\textheight}{39\baselineskip}
\addtolength{\textheight}{\topskip}
\setlength{\voffset}{-0.5in}
\setlength{\headsep}{0.3in}
%
\newcommand{\divergence}{\mathrm{div}\,}  %ダイバージェンス
\newcommand{\grad}{\mathrm{grad}\,}  %グラディエント
\newcommand{\rot}{\mathrm{rot}\,}  %ローテーション
%
\pagestyle{myheadings}
\markright{\footnotesize \sf ベイズ推定勉強会}
\begin{document}
%
%
\section*{問1}
ある事象Aがおこった元で事象Bが起こる確率$ P(A  \mid  B) $\\
 例題\\
 1組のトランプから1枚抜くとする.抜いたカードがハートである事象をA,絵札である事象Bとしたとき,条件付き確率$ P(A  \mid  B) $ ,$ P(B  \mid A ) $ ,同時確率$  P(A \cap B) $を求めよ.
\section*{問2}
形が同じの2つの壺,赤い壺,青い壺,が置いてある.
赤い壺には白玉2個,赤玉3個が入っていて,青い壺には,白玉4個,赤玉8個入っている.
壺を一つを選択し,玉を一つを取り出したら,白玉だった.このときの赤い壺から取り出された確率を求めよ.
\section*{問3}
A社の壺には,水晶玉とガラス玉が4:1の割合で入っている.
B社の壺には,水晶玉とガラス玉が2:3の割合で入っている.壺の見た目は全く同じで壺にはたくさん玉が入っているとする.
いま,A社製かB社製か不明の壺があり,続けて3回玉を取り出したら,順に 水晶玉,水晶玉,ガラス玉であった.この壺がA社製の確率を求めよ.
\section*{問4}
ある工場から作られるチョコレート菓子の内容量は正規分布に従い分散は$ 1^2$であり,製品の1つを選んで調べたら,その内容量は101gであった.この時のこの工場から作られる製品内容量の「平均値 $\mu$を求めよ」
\end{document}